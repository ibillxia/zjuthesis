\input{./Chapters/chap3_1}

\input{./Chapters/chap3_2}

\section{插入算法伪代码}
以下是插入算法伪代码示例,如算法\ref{alg:skeleton_dt}所示。

\begin{algorithm}[!htb]
\caption{基于距离变换的骨架提取}
\label{alg:skeleton_dt}
\begin{algorithmic}[1]
\Require 前景的二值图 bw \Comment{像素的灰度值为0或1} 
\Ensure 骨架图 skel
\State // 第1次遍历:从上往下,从左往右
\For{$i=1,\dots ,M$} \Comment{M是二值图的高度}
    \For{$j=1,\dots ,N$} \Comment{N是二值图的宽度}
        \State bw[i][j] = 1 + min(bw[i][j-1], bw[i-1][j])  \Comment{min函数取极小值} 
    \EndFor
\EndFor
\State // 第2次遍历:从下往上,从右往左
\For{$i=M,\dots ,1$}
    \For{$j=N,\dots ,1$}
        \State bw[i][j] = 1 + min(bw[i][j], bw[i+1][j], bw[i][j+1])
    \EndFor
\EndFor
\State // 第3次遍历:获取骨架图
\State skel 的空间分配,并将每个像素初始化为0
\For{$i=1,\dots ,M$}
    \For{$j=1,\dots ,N$}
        \State t = max(bw[i-1][j], bw[i+1][j], bw[i][j-1], bw[i][j+1])  \Comment{max函数取极大值} 
        \State t = max(t, bw[i-1][j-1],bw[i-1][j+1],bw[i+1][j-1],bw[i+1][j+1])
        \If {bw[i][j]>=t}
        	\State skel[i][j]=1 \Comment{骨架点} 
        \Else 
        	\State skel[i][j]=0
        \EndIf
    \EndFor
\EndFor
\end{algorithmic}
\end{algorithm}
